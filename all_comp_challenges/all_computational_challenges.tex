\documentclass[12pt]{article}
 \usepackage[letterpaper, left=2cm, right=2cm, height=27cm,includeheadfoot]{geometry}
\usepackage{setspace}
\usepackage{latexsym}
\usepackage{amsmath}

\newcommand{\bs}{\boldsymbol}
\newcommand{\Xb}{\bs{X}}
\newcommand{\Bb}{\bs{B}}
\newcommand{\Ab}{\bs{A}}
\newcommand{\ab}{\bs{a}}
\newcommand{\bb}{\bs{b}}
\newcommand{\zb}{\bs{z}}
\newcommand{\Yb}{\bs{Y}}
\newcommand{\yb}{\bs{y}}
\newcommand{\xb}{\bs{x}}
\newcommand{\vb}{\bs{v}}
\newcommand{\mub}{\bs{\mu}}
\newcommand{\Sigb}{\bs{\Sigma}}
\newcommand{\Ib}{\bs{I}}
\newcommand{\Jb}{\bs{J}}
\newcommand{\Pb}{\bs{P}}
\newcommand{\oneb}{\bs{1}}
\newcommand{\Nr}{\text{N}}
\newcommand{\bbeta}{\bs{\beta}}
\newcommand{\btheta}{\bs{\theta}}
\newcommand{\eb}{\bs{\epsilon}}
\newcommand{\zerob}{\bs{0}}
\newcommand{\stc}{\stackrel}
\usepackage{multirow,array}

\usepackage{hyperref}

\begin{document}

\begin{center}
{\Huge \bf Stochastic Models in Biology: Computational Challenges
\\ \vskip5mm  }
\end{center}

By: Kre\v simir Josi\`c and Alexander Stewart.

\begin{center}
{\Large \bf Computational Challenge 0
\\ \vskip5mm  }
\end{center}

As discussed in class, write a short Python program that takes a sequence of 19 numbers, and 
generates 4 groups randomly. There should be 3 groups of 5, and one group of 4 numbers.

Save the code in your GitHub repository and send me the link.  I will then clone the repository, and
make sure that everything works.  You will also use the repository to turn in homeworks, so 
it is good to set it up early.


\begin{center}
{\Large \bf Computational Challenge 1
\\ \vskip5mm }
\end{center}

Rejection sampling is a  useful method for generating random samples from a distribution $f(x)$ that 
may be hard to sample from.  The idea is to generate samples from a known distribution $g(x)$, and
then either to accept or reject the samples.  The criterion for acceptance and rejection,  and the algorithm are described here:

\vskip5mm

\href{https://bookdown.org/rdpeng/advstatcomp/rejection-sampling.html}{https://bookdown.org/rdpeng/advstatcomp/rejection-sampling.html}

\vskip5mm

Use either the algorithm described in sections 6.3.1 or 6.3.3 on this webpage to do the following

\begin{enumerate}
\item[\bf Group 1] Use the uniform density on the unit interval as a candidate density to generate samples from the a beta distribution.
\item[\bf Group 2] Use the normal distribution as a candidate density to generate samples from the $t$-distribution.
\item[\bf Group 3] Use the Poisson distribution as a candidate distribution to generate samples from the binomial distribution.
\item[\bf Group 4] Use the normal distribution as a candidate distribution to generate samples from a probability distribution composed of the sum of three other normal distributions with means $\mu_1=-1$, $\mu_2=0$ and $\mu_3=1$ and standard deviations $\sigma_1=\sigma_2=\sigma_3=0.5$.
\end{enumerate}

Questions for discussion:
\begin{enumerate}
\item When do you think the method becomes inefficient?  
\item Do you think that rejection sampling can be extended to higher dimensions? How?
\end{enumerate}

\vskip5mm 


\begin{center}
{\Large \bf Computational Challenge 2
\\ \vskip5mm }
\end{center}


In this computational challenge you will implement one of two Markov chain models of a biological system:
\\
\\
\textbf{Model 1: Quasi-stationary distributions in the birth-death model.}  In a general birth and death process, the probability that the 
population increases or decreases between steps can depend on the population size.  Denote the
size of a population at time $t$ by $X_t$. The transition probabilities are then given by:
\begin{align*}
P(X_{t+1} &= j+1 | X_{t} = j) = b_j \\
P(X_{t+1} &= j-1 | X_{t} = j) = d_j \\
P(X_{t+1} &= j | X_{t} = j) = 1 - ( b_j + d_j) 
\end{align*}
where additionally $d_0 = 0$ and $b_ 0 = 0$, so that a population with no individuals is an absorbing state.  Moreover,
$\text{sup}_i ( b_i + d_i) \leq 1$.  Note that all states, except for 0, are transient.  To keep the population finite we 
also assume that $b_N = 0$ for some $N > 1$.

If you don't remember the logistic growth model covered in most calculus classes, refresh your memory by reading the relevant Wikipedia page, or
consulting any other resource. For a logistic growth process we can assume $b_i - d_i = ri (1 - i/K)$. Note that any choices of $b_i$ and $d_i$
that satisfy this equation are consistent with logistic growth because the expected increment over a time interval is $ri (1 - i/K)$.

Assume that $b_i = r(i - i^2/2K)$ and $d_i = r i^2/2K$, so that the maximal population size is $N = 2K$ (otherwise transition probabilities become negative), 
and the carrying capacity of the environment is $K$.
\begin{itemize}
\item Implement computationally the Markov chain defined by these transition probabilities.  Choose $r = 0.004$, $K = 50$.  Plot several trajectories to get
a feel for the behavior of the population.  After several thousand timesteps the population size should fluctuate around 50.  Run the simulation 
up to $t = 3000$ (or a time of your choices where the mean stabilizes) a number of times to create a histogram of the population size at that time. 
\item Set $r = 0.015$ and $K = 8$.  Run the simulation until the population goes extinct starting at different initial population sizes.  Plot the mean 
time to extinction.
\item Discuss how the answers to the previous two questions can be consistent: The first suggests that the population reaches a stable distribution
around the carrying capacity.  The second suggests that the population goes extinct.
\end{itemize}


\noindent \textbf{Model 2: Fixation in the Moran model.}  The Moran model describes the frequency of different genes in a population of fixed size $N$.
Here we will consider a haploid population with a genotype composed of a single locus, which can be occupied by one of two alleles, $a$ and $A$. We will assume that both alleles are present in the population, and that there is no selection for or against either allele.

Denote the number of individuals with allele $a$ by $j$, and the number of individuals with allele $A$ by $N-j$. The Moran model assumes that birth and death occur simultaneously, as a single event. First an individual is chosen to reproduce with mutation. A second individual is then chosen to die. This birth-death event constitutes a single step in the Moran process. Denote the
number of individuals with allele $a$ at time $t$ by $X_t$. The transition probabilities for this Moran process are given by:

\begin{align*}
P(X_{t+1} &= j+1 | X_{t} = j) = \left(\frac{j}{N}(1-\mu_a)+\frac{N-j}{N}\mu_A\right)\times\left(\frac{N-j}{N}\right) \\
P(X_{t+1} &= j-1 | X_{t} = j) =  \left(\frac{N-j}{N}(1-\mu_A)+\frac{j}{N}\mu_a\right)\times\left(\frac{j}{N}\right) \\
P(X_{t+1} &= j | X_{t} = j) = 1 -  \left(\frac{j}{N}(1-\mu_a)+\frac{N-j}{N}\mu_A\right)\times\left(\frac{N-j}{N}\right)- \left(\frac{N-j}{N}(1-\mu_A)+\frac{j}{N}\mu_a\right)\times\left(\frac{j}{N}\right)
\end{align*}

where $\mu_a$ is the probability that allele $a$ mutates to allele $A$ during reproduction and $\mu_A$ is the probability that allele $A$ mutates to allele $a$.   

\begin{itemize}
\item Assume that $\mu_a=\mu_A=0$. In this case the Moran model has two absorbing states at $j=0$ and $j=N$, i.e. one state in which all individuals have allele $A$ and one in which all individuals have allele $a$. Numerically estimate the probability that the Moran process initialized at $X_0=1$ will eventually reach fixation for $a$, i.e. the absorbing state in which all individuals have allele $a$.
\item Assume again that $\mu_a=\mu_A=0$. Conditional on $a$ reaching fixation, estimate the time for $a$ to become fixed for a Moran process initialized at $X_0=1$.
\item Numerically estimate the stationary distribution for the Moran process with (i) $\mu_a=\mu_A=0.1/N$,  (ii) $\mu_a=\mu_A=1/N$ and  (iii) $\mu_a=\mu_A=10/N$.
\item Briefly discuss how adding selection for and against $a$ might change these results.
\end{itemize}

\begin{enumerate}
\item[\bf Group 1] Implement model 1.
\item[\bf Group 2] Implement model 1.
\item[\bf Group 3] Implement model 2.
\item[\bf Group 4] Implement model 2.
\end{enumerate}


\begin{center}
{\Large \bf Computational Challenge 3
\\ \vskip5mm  }
\end{center}

In this challenge you will implement stochastic simulations of two synthetic biological circuits.
The synthetic genetic switch, and the repressilator were described in two back--to--back papers
in \emph{Nature} in 2000.  Please follow the link on the website to the two papers. 




\begin{center}
{\Large \bf Computational Challenge 4
\\ \vskip5mm }
\end{center}

In this challenge you will implement stochastic simulations of two synthetic biological circuits.
The synthetic genetic switch, and the repressilator were described in two back--to--back papers
in \emph{Nature} in 2000.  Please follow the link on the website to the two papers. 

In this challenge two groups will report on the deterministic version of models of these circuits, while the other two
will report on the stochastic versions. The last part of each question asks that you 
compare the results of the simulations. Therefore groups 1 and 2, and groups 3 and 4 will
need to work together on each project.


%
\begin{enumerate}
\item[\bf Group 1] You will analyze the deterministic version of the genetic switch system described in section III of the paper by Loinger, et al. 
(the one discussing the stochastic genetic switch).  The differential equation is given in Eq. (9). 
\begin{itemize}
\item Construct a Petri net representation of the system based on the description in the text  (you can
also consider the master equation given in the same section).   
How does the ODE relate to this representation? Why is this called an exclusive switch?
\item Solve the ODE numerically and show the different trajectories in the phase plane.
Coordinate with Group 2 on the parameters -- use the ones in the paper.  
The nullclines are the solution to the equations $A' = 0$ and $B' = 0$.  As discussed in the paper
by Gardner, et al. they determine the behavior of the deterministic system (See Fig. 2). Perform the equivalent
analysis for the system in Eq. (9).
\end{itemize}



\item[\bf Group 2] Consider the exclusive switch discussed in section III of the paper by Loinger, et al. 
(the one discussing the stochastic genetic switch). The master equation for this system is given in Eq. (14) 
\begin{itemize}
\item Construct a Petri net representation of the system based on the description in the text 
and the master equation. 
\item  Set up a stochastic simulation of this system. Use the parameters discussed in the text, 
and change $k$ from 0.005 to 50, as in figure 4. What happens to the switching times 
between the two states as $k$ is increased? You can use the Gillespie algorithm. 
Show the traces for $N_{A}$ and $N_{B}$ in three or four cases.   Discuss whether the
change is as expected.
\item Together with Group 1 explain whether or whether not these results can be explained 
using the deterministic equations.
\end{itemize}

\item[\bf Group 3] Consider the deterministic repressilator described via the Michaelis-Menten equations in Loinger, et al (the one discussing the repressilator).
\begin{itemize}
\item Construct a Petri net representation of the system for case (i) with mRNA (i.e. corresponding to Eq. 1) and case (ii) without mRNA (i.e. corresponding to Eq.2)
\item  Solve the ODEs for case (i) numerically with parameters $k=d_m=1$ and $d_p=g_p$. Plot phase portraits showing the region in which the system displays oscillations, with values of $g_p$ on the x-axis plotted against values of $g_m$ on the y-axis, for Hill coefficients $n=[1,2,3]$ (i.e three different phase plots, one for each value of $n$). Repeat the same exercise for case (ii) 
\end{itemize}

\item[\bf Group 4] Consider the stochastic repressilator described in Loinger, et al (the one discussing the repressilator) and in Elowitz and Leibler.
\begin{itemize}
\item Using both repressilator papers as a guide, write down a Master equation describing the stochastic version of Eq. 1 of Loinger (i.e a simple repressilator with reactions as described in the Michaelis-Menten kinetics section of that paper).
\item Suggest a method for detecting oscillation-like behavior in a stochastic process. Apply your method to stochastic simulations of the simple repressilator for some of the same cases considered by Group 3 above (i.e., you do not need attempt a full phase portrait, but choose a few parameter sets that correspond to those explored by group 3, and try to find at least one case where oscillations appear to be happening).
\end{itemize}

\end{enumerate}


\begin{center}
{\Large \bf Computational Challenge 5
\\ \vskip5mm }
\end{center}


In this challenge two groups will report on the deterministic version of models of these circuits, while the other two
will report on the stochastic versions. The last part of each question asks that you 
compare the results of the simulations. Therefore groups 1 and 2, and groups 3 and 4 will
need to work together on each project.


%
\begin{enumerate}
\item[\bf Group 1] Genetic circuits that oscillate with a period of approximately 24 hours are widespread across different species.  The details of the oscillators  vary considerably between species, but  the ones have been characterized rely on some for of negative auto-regulation. The number of molecules involved in these oscillations also varies greatly between species. 

One of the most influential models of circadian clocks is the Goodwin Oscillator.  It takes the form 
\begin{eqnarray*}
X' &=& \alpha_1 \frac{K^N}{K^N + Z^N} - \gamma_1 X \\
Y' &= & \alpha_2 X - \gamma_2 Y \\
Z' &=& \alpha_3 Y - \gamma_3 Z \\
\end{eqnarray*}

\begin{enumerate}
\item Implement the oscillator by solving the ODEs for the values in the paper by Gonze and Ruoff (see Fig. 2).  Show the trajectories for all the three species.
\item Work with Group 2 to write a corresponding Petri Net.  Assume that the equations are given for concentrations. What are reasonable values for the reaction rates so that the molecule number stays within at most a few hundred molecules in the SSA (to be implemented by Group 2)?  
\item Describe the negative auto-regulation intuitively. What is the main negative auto-regulation term?  Where does the delay come from and why is it important?
\item Change the Hill coefficient, $n,$ to a value smaller than 8. What happens to the oscillations?  Why do you think that this happens.
\end{enumerate}

\item[\bf Group 2] 
First work with Group 1 to understand the oscillations in this circuit, and on part 2 of their problem. Once you have the reactions 
and candidate rates do the following:
\begin{enumerate}
\item Implement a stochastic simulation corresponding to the system above using the Gillespie Algorithm.  You may need to 
adjust the rates to get oscillations, and keep the number of molecules at a reasonable level (more than 10 or so, and less than a few hundred).
\item Change the Hill coefficient, $n$ to a value smaller than 8. What happens to the oscillations?  Send a message to Group 1 to see
what they observed, and compare notes.
\item Compare the frequency of oscillations with those obtained by Group 1.  
\end{enumerate}

\item[\bf Group 3] A feed-forward loop (FFL) is a 3-gene genetic circuit in which two ``input'' transcription factors (TFs), $X$ and $Y$, regulate a third ``target'' $Z$, with TF $X$ also regulating TF $Y$. Depending on the pattern of activation and repression among the three genes, an FFL can speed or slow the response time of $Z$ to a stimulus in the form of $X$ switching ON or OFF. Here we will explore the behavor of the so-called ``incoherent'' FFL.
\\
\\
We will assume that $X$ is either OFF or ON such that the concentration of the product of gene $X$ is $X^*$ where $X^*=1$ when the gene is ON and $X^*=0$ when the gene is OFF. We will look at two specific cases of an incoherent FFL.
\\
\\
CASE I: In the first case, the concentrations of gene product $Y$ and $Z$ are described by the following set of ODEs:

\begin{eqnarray*}
\frac{dY}{dt}&=&B_y+\beta_y\frac{\left(X^*/K_{xy}\right)^h}{1+\left(X^*/K_{xy}\right)^h} -\alpha_yY\\
\frac{dZ}{dt}&=&B_z+\beta_z\frac{\left(X^*/K_{xz}\right)^h}{1+\left(X^*/K_{xz}\right)^h}\times\frac{1}{1+\left(Y/K_{yz}\right)^h} -\alpha_z Z\\
\end{eqnarray*}
\\
CASE II: In the second case, the concentrations of gene product $Y$ and $Z$ are described by the following set of ODEs:

\begin{eqnarray*}
\frac{dY}{dt}&=&B_y+\beta_y\frac{1}{1+\left(X^*/K_{xy}\right)^h} -\alpha_yY\\
\frac{dZ}{dt}&=&B_z+\beta_z\frac{1}{1+\left(X^*/K_{xz}\right)^h} \times \frac{1}{1+\left(Y/K_{yz}\right)^h} -\alpha_z Z\\
\end{eqnarray*}

\begin{enumerate}
\item Draw network diagrams showing genes as nodes and regulatory interactions as edges for CASE I and CASE II. Label the edges as either activation or repression.
\item Use the parameters $B_y=B_z=0$, $\alpha_y=\alpha_z=1$, $\beta_y=\beta_z=1$, $K_{xz}=K_{xy}=0.1$ and $h=2$. Consider the FFL in CASE I. Compare the behavior of the FFL with $K_{yz}=0.01$ (strong $Y-Z$ regulation) to the same FFL with $K_{yz}=10$ (very weak $Y-Z$ regulation) in response to $X$ switching from ON to OFF. Repeat the same exercise for the response to $X$ switching from OFF to ON. What do you observe about the response times of $Z$ in the two cases?
\item Repeat the same exercise for the FFL described in CASE II.
\item Summarize your results. When might the two different circuits be useful in a biological setting (i.e. what might an organism use them for)?
\end{enumerate}


\item[\bf Group 4] Work with Group 3 to understand the behavior of the different circuits in response to changes in the input $X$.

\begin{enumerate}
\item Construct a Petri Net describing a stochastic version of the two FFLs. Assuming the ODEs describe concentrations, suggest reasonable values for reaction rates that will produce at most a few hundred molecules in the products of all three genes, while retaining the regulatory relationships of the deterministic system.
\item Implement a stochastic simulation corresponding to the two FFLs above using the Gillespie Algorithm. 
\item Estimate the ON-to-OFF and OFF-to-ON response times with strong $Y-Z$ regulation for the two FFL cases.
\item Assuming $X$ is ON, plot the stationary distribution for the expression of $Z$ with strong $Y-Z$ regulation for the two FFL cases.

\end{enumerate}

\end{enumerate}


\clearpage


\begin{center}
{\Large \bf Computational Challenge 6
\\ \vskip5mm }
\end{center}



In this challenge you will examine 3 different version of the integrate--and--fire model: the leaky (QIF), the quadratic,
and the exponential integrate-and-fire model. The fourth group will implement an adaptive version of the integrate and fire model.

To start, you will need a function that generates an inhomogeneous Poisson spike train. You can do this by using
the method that we described in an earlier computational challenge, or simply using discrete increments.  Use a resolution
of at least 0.5 \emph{ms}.   

\begin{enumerate}
\item[\bf Group 1]  The leaky integrate--and--fire neuron has the form
$$
\tau_m u' = - u + R I(t).
$$
Set $\tau_m = 10 ms$ and $R = 500M \Omega$.  Set the threshold at 20 $mV$. Remember that the voltage needs to be reset once the threshold is reached.  You do not need to use a refractory period.   

\begin{enumerate}
\item Assume the input is described by a Poisson process. Compute the firing rate as a function of the input rate, assuming that the weight of each synapse is $w = 1$,
with current measured in $pA$.  Assume that each input spike results in an instant depolarization.  As a result model the effect of each spike as  an instantaneous increase in the membrane potential
by $\Delta u = 0.5mV$.  You can compute the output rate by recording the times between consecutive spikes, and dividing by the total
number of spikes.
\item Assume that the input is composed of two independent Poisson spike trains.  One corresponds to inputs with synapse weights $w = 1$ (excitatory inputs), and the other with $w = -1$ (inhibitory inputs).  Denote the two input rates by $\nu_E$ and $\nu_I$. Start with a $\nu_E$ that gave an output spike rate of 20$Hz$ in the previous part. Gradually increase the $\nu_I$, and plot the output firing rate, and the coefficient of variation for the inter-spike interval (time between spikes).  Make sure to plot some output spike trains. You can represent them as raster plots.
\end{enumerate}

\item[\bf Group 2] 
The exponential integrate and fire (EIF) model is given by 
\begin{equation} \label{E:EIF}
\tau_m u' = - u + \Delta_T \exp \left( \frac{u - \eta_{\text{rh}}}{\Delta_T} \right) + R I(t) 
\end{equation}
Set $\tau_m = 10 ms$ and $R = 500M \Omega$. Remember that the voltage needs to be reset once the threshold is reached.  You do not need to use a refractory period.   

\begin{enumerate}
\item Set $\eta_{\text{rh}} = 18 mV$ and $\Delta_T = 1 mV$. Plot the right hand side of Eq.~\eqref{E:EIF} and vary the input current (measure the input
current in $pA$).  What happens to the roots of the equation $- u + \Delta_T \exp \left( \frac{u - \eta_{\text{rh}}}{\Delta_T} \right) + R I(t) = 0$,
as $I$ is increased from 0? 
\item Simulate the EIF for different values of constant input current, as you pass the bifurcation point observed in the previous problem. 
You will have to set the threshold around $50mV$.  Be careful since the voltage can grow explosively.  Make sure to plot the membrane potential.
\item Assume the input is described by a Poisson process. Compute the firing rate as a function of the input rate, assuming that the weight of each synapse is $w_k = 1$,
with current measured in $pA$.  Assume that each input spike results in an instant depolarization.  As a result model the effect of each spike as  an instantaneous increase in the membrane potential
by $\Delta u = 0.5mV$.  You can compute the output rate by recording the times between consecutive spikes, and dividing by the total
number of spikes. Do this for $\Delta_T = 1, 0.5,$ and $0.1$.
\end{enumerate}

\item[\bf Group 3] 
The quadratic integrate and fire (QIF) model is given by 
\begin{equation} \label{E:QIF}
\tau_m u' = - a_0 (u - u_{\text{rest}}))(u - u_c) + R I(t) 
\end{equation}
Set $\tau_m = 10 ms$ and $R = 500M \Omega$. Remember that the voltage needs to be reset once the threshold is reached.  You do not need to use a refractory period.   

\begin{enumerate}
\item Set $u_{\text{rest}}  = 0 mV$ and $u_c = 20mV$. Plot the right hand side of Eq.~\eqref{E:QIF} and vary the input current (measure the input
current in $pA$).  What happens to the roots of the equation $- a_0 (u - u_{\text{rest}}))(u - u_c) + R I(t)  = 0$,
as $I$ is increased from 0? 
\item Simulate the QIF for different values of constant input current, as you pass the bifurcation point observed in the previous problem. 
You will have to set the threshold around $60mV$.  Be careful since the voltage can grow quickly.
\item Assume the input is described by a Poisson process. Compute the firing rate as a function of the input rate, assuming that the weight of each synapse is $w_k = 1$,
with current measured in $pA$.  Assume that each input spike results in an instant depolarization.  As a result model the effect of each spike as  an instantaneous increase in the membrane potential
by $\Delta u = 0.5mV$.  You can compute the output rate by recording the times between consecutive spikes, and dividing by the total
number of spikes.  Do this for $a_0 = 1, 2$, and $5$.
\end{enumerate}

\item[\bf Group 4] 
The leaky integrate and fire model with an adaptive current has the form 
\begin{equation} \label{E:AdEIF}
\begin{split}
\tau_m u' &= - u  + R I(t)  - R v \\
\tau_v v' & = a u -v  + b \tau_v \sum_{t^f} \delta(t - t^f) 
\end{split}
 \end{equation}
The sum $\sum_{t^f} \delta(t - t^f)$ goes over all spike times (threshold crossing times) of the model.
 Set $\tau_v = 100ms$, $\tau_m = 10 ms,$ and $R = 500M \Omega$.  Set the threshold at 20 $mV$. 
 You may find Chapter 6 in the book by Gerstner, et al. helpful for this problem.

\begin{enumerate}
\item Set $ b = 5 pA$, and vary $a$ from negative to positive values. Plot some of the firing patters that you observe as you increase the input current $I$ 
from zero to 100$pA$. 
\item Assume the input is described by a Poisson process. Assume that the weight of each synapse is $w = 1$,
with current measured in $pA$.  Assume that each input spike results in an instant depolarization.  As a result model the effect of each spike as  an instantaneous increase in the membrane potential
by $\Delta u = 0.5mV$.  Take two different output patterns you observed in the previous part, and see if they are changed if the input is described by a Poisson process, with the same average that you used in part (a).
\item Assume that the input is composed of two independent Poisson spike trains.  One corresponds to inputs with synapse weights $w = 1$ (excitatory inputs), and the other with $w = -1$ (inhibitory inputs).  Denote the two input rates by $\nu_E$ and $\nu_I$. Start with a $\nu_E$ that gave an output spike rate of 20$Hz$ in the previous part with an adaptive output. Gradually increase the $\nu_I$, and plot the output firing rate, and the coefficient of variation for the inter-spike interval (time between spikes).  Make sure to plot some output spike trains. You can represent them as raster plots.  Discuss with group 1 the effect of the adaptive current on the output.
\end{enumerate}

\end{enumerate}


\begin{center}
{\Large \bf Computational Challenge 7
\\ \vskip5mm }
\end{center}

In this challenge you will examine the behavior of the Wright-Fisher and Moran process at different population sizes   

\begin{enumerate}
\item[\bf Group 1]  The haploid Wright-Fisher model with a single locus and two alleles $a$ and $A$ has has transition probabilities

\begin{eqnarray*}
p_{i,j}={N \choose j}\left((1-u) \frac{w_a i}{w_a i+(N-i)w_A}+v\frac{w_A (N-i)}{w_a i+(N-i)w_A}\right)^j\times\\
\left((1-v) \frac{w_A(N- i)}{w_a i+(N-i)w_A}+u\frac{w_a i}{w_a i+(N-i)w_A}\right)^{N-j}
\end{eqnarray*}
\\ 
where the terms are as defined in lecture 14 and 15.

\begin{enumerate}
\item Setting $w_a=1+s$ and $w_A=1$, find the stationary distribution of the Wright-Fisher process at population sizes of $N=10$, $N=100$, $N=1000$ and $N=10000$. You should work with the other three groups and choose appropriate values of $u$, $v$ and $s$. Note these parameters should be chosen so that the values of $Nu$, $Nv$ and $Ns$ are the same for each population size [e.g. if you choose $v=0.01$ for $N=10$ you must use $v=0.001$ when $v=100$ and so on]. Compare the stationary distribution produced via simulation to that given in the lectures as the solution for the Fokker Planck equation $P^*=x^{Nv-1}(1-x)^{Nu-1}e^{Nsx}$

\item Now set mutation rates $u=v=0$. Beginning with a single copy of allele $a$, i.e. $i=1$ use simulation to estimate the probability of fixation of $a$ for the same population sizes and values of $s$ as used previously. Compare the results to the analytic expression derived for the Moran model $f_1=\frac{w_A/w_a-1}{(w_A/w_a)^N-1}$ and to the approximation commonly quoted for the Wright-Fisher model $f_1=(1-e^{-s})/(1-e^{-Ns})$. 

\item In both cases explain any differences you see between the simulated and analytical results.

\end{enumerate}

\item[\bf Group 2] 
The haploid Moran model with a single locus and two alleles $a$ and $A$ has has transition probabilities

\begin{eqnarray*}
T(n+1|n)&=(1-u)\frac{w_a i}{w_a i+(N-i)w_A}\left(\frac{n}{N}\right)\left(1-\frac{n}{N}\right)+v\frac{w_A (N-i)}{w_a i+(N-i)w_A}\left(1-\frac{n}{N}\right)^2\\
T(n-1|n)&=(1-v)\frac{w_A (N-i)}{w_a i+(N-i)w_A}\left(1-\frac{n}{N}\right)\left(\frac{n}{N}\right)+u\frac{w_a i}{w_a i+(N-i)w_A}\left(\frac{n}{N}\right)^2\\
\end{eqnarray*}
\\

\begin{enumerate}
\item Using the parameter values you agreed with group 1, repeat the same tasks, (a), (b) and (c), set for the Wright-Fisher model above for the Moran model.
\end{enumerate}

\item[\bf Group 3] 
You will implement \emph{both} the Wright Fisher and Moran models described above in a scenario in which the population is split between two islands, such that $M$ individuals are present on island 1 and $N-M$ individuals on island 2. The fitness of alleles $a$ and $A$ are assumed to be the same on both islands but, the fitness of an individual is calculated relative tot eh other members of the island only. Migration between the islands is assumed to occur at rate $m_{12}$ from island 1 to 2 and $m_{21}$ in the other direction.

\begin{enumerate}
\item Write down the transition probabilities for for the Wright Fisher and Moran models in this two island setting
\item Using the parameter values agreed with group 1 for $u$, $v$ and $s$, and using $N=10000$, repeat the same tasks as set for group 1 (i.e. find the stationary distribution with mutation and fixation probability without mutation) for both models, for values of $M=50$, $M=500$ and $M=5000$. 
\item Discuss how the island structure impacts evolution.
\end{enumerate}

\item[\bf Group 4] 
You will implement the Moran model in a scenario in which allele $a$ is frequency dependent, meaning $w_a=1+(i-1)s/N$ and $w_A=1+is/N$.


\begin{enumerate}
\item Write down the transition probabilities for for the Moran model with frequency dependent fitness in this two island setting described for group 3.
\item Using the parameter values agreed with group 1 for $u$, $v$ and $s$, and using $N=10000$, repeat the same tasks as set for group 1 (i.e. find the stationary distribution with mutation and fixation probability without mutation) for both models, for values of $M=0$ (i.e. only one island) and for $M=50$, $M=500$ and $M=5000$. 
\item Discuss how this frequency dependence impacts evolution.
\end{enumerate}

\end{enumerate}



\begin{center}
{\Large \bf Computational Challenge 8
\\ \vskip5mm  }
\end{center}

In this challenge you will examine games played in evolving populations.

\begin{enumerate}
\item[\bf Group 1]  Consider the one-shot rock-paper-scissors game played between two players $X$ and $Y$. The game has payoff matrix:

\begin{table}[h]
    \setlength{\extrarowheight}{2pt}
    \begin{tabular}{ccc|c|cl}
      & \multicolumn{1}{c}{} & \multicolumn{3}{c}{Player $Y$}\\
      & \multicolumn{1}{c}{} & \multicolumn{1}{c}{$R$}  & \multicolumn{1}{c}{$P$}  & \multicolumn{1}{c}{$S$} \\
      \multirow{3}*{Player $X$}  & $R$ & $(0,0)$ & $(-1,1)$  & $(1,-1)$  \\\cline{3-5}
      & $P$ & $(1,-1)$ & $(0,0)$ & $(-1,1)$ \\\cline{3-5}
      & $S$ & $(-1,1)$ & $(1,-1)$ & $(0,0)$ \\
    \end{tabular}
  \end{table}

Consider a strategy $\mathbf{p}=\{p_{r}, p_p,p_{s}\}$ where $p_r$ is the probability of playing paper etc. and $p_r+p_s+p_p=1$.

\begin{enumerate}
\item Write down the expected payoff for a pair of strategies $\mathbf{p}$ and $\mathbf{q}$ in the one-shot rock-paper-scissors game. Can you find the Nash equilibrium for this system?
\item We are now going to model the evolution of rock-paper-scissors strategies. Assume there are only three strategies, always rock, always paper or always scissors, i.e. $p_r=1$, $p_s=1$ or $p_p=1$. Write down the fitness for each of these three strategies under the Moran model.
\item Investigate via simulation the time evolution of the strategies in a population of size $N=1000$. Assume that strategies can mutate to one another at rate $\mu=0.001$ with $w=1$. Plot a time series for the frequency of different strategies in the system. Also plot a histogram showing the frequency distribution for each strategy once the system has reached equilibrium [note you will have to decide when to stop your simulation in order to do this].

\end{enumerate}

\item[\bf Group 2] 
Consider the one-shot Hawk-Dove game played between two players $X$ and $Y$:

\begin{table}[h]
    \setlength{\extrarowheight}{2pt}
    \begin{tabular}{ccc|cc}
      & \multicolumn{1}{c}{} & \multicolumn{2}{c}{Player $Y$}\\
      & \multicolumn{1}{c}{} & \multicolumn{1}{c}{$H$}  & \multicolumn{1}{c}{$D$}  \\
      \multirow{2}*{Player $X$}  & $H$ & $((B-C)/2,(B-C)/2)$ & $(B,0)$   \\\cline{3-4}
      & $D$ & $(0,B)$ & $(B/2,B/2)$ & \\
    \end{tabular}
  \end{table}
  
Consider a strategy $p$ which gives the probability of playing hawk in the one-shot game. The probability of playing dove is thus $1-p$.  

\begin{enumerate}
\item Write down the expected payoff for a pair of strategies $p$ and $q$ in the one-shot Hawk-Dove game. Can you you find the Nash equilibrium for this system?
\item We are going to study the evolution of hawks and doves under the Moran model. We will assume that each individual is either a hawk $p=1$ or a dove $p=0$. Write down the fitness of hawks and doves under the Moran model.
\item Investigate via simulation the equilibrium frequency of hawks and doves in a population on $N=1000$, assuming mutations between strategies at rate $\mu=0.001$ with $w=1$. For $B=4$ and $C=6$ show a histogram for the distribution of strategies at equilibrium. Show how the average proportion of hawks and doves changes as you change $B$.
\end{enumerate}

\item[\bf Group 3] 
Consider a one-shot, two-player Public Goods game played between two players $X$ and $Y$. This game is similar to the donation game discussed in class and the moves are cooperate or defect:

\begin{table}[h]
    \setlength{\extrarowheight}{2pt}
    \begin{tabular}{ccc|cc}
      & \multicolumn{1}{c}{} & \multicolumn{2}{c}{Player $Y$}\\
      & \multicolumn{1}{c}{} & \multicolumn{1}{c}{$C$}  & \multicolumn{1}{c}{$D$}  \\
      \multirow{2}*{Player $X$}  & $C$ & $(B-C,B-C)$ & $(B/2-C,B/2)$   \\\cline{3-4}
      & $D$ & $(B/2,B/2-C)$ & $(0,0)$ & \\
    \end{tabular}
  \end{table}

\begin{enumerate}
\item When is this game a Prisoner's Dilemma? What is the Nash equilibrium for the one-shot game when it is not a Prisoner's Dilemma?
\item Now consider the iterated version of the game, in which both players use memory-1 strategies. Consider the strategies ALLD, $\{0,0,0,0\}$, TFT $\{1,0,1,0\}$, ALLC $\{1,1,1,1\}$ and WSLS $\{1,0,0,1\}$. Assuming there is a small error rate $\epsilon=0.001$ such that a player who attempts to play cooperate accidentally plays defect and vice versa, determine and plot the stationary distributions for the iterated game played between all pairs of strategies (ie. 10 different matchups) 
\item Using these stationary distributions calculate the expected payoffs for all pairs of strategies in the infinitely iterated game for $B=1.5$ and $C=1$.
\end{enumerate}

\item[\bf Group 4] 
Consider the two-player public goods game described above for Group 3


\begin{enumerate}
\item We will consider three iterated prisoner's dilemma strategies in the absence of errors, ALLD, $\{0,0,0,0\}$, TFT $\{1,0,1,0\}$ and ALLC $\{1,1,1,1\}$. Assuming TFT starts off cooperating, and there are no errors, write down the payoffs received for each of the six pairwise matchups between these strategies when playing against the other in the infinitely iterated game (you should not need to simulate this)
\item Write down the fitness for each of these strategies under the Moran model. Calculate via simulation the fixation probability for each of the possible invasions (6 in total), for a population of $N=1000$ and $w=1$.
\item Now we will run simulations allowing mutations between strategies at rate $\mu=0.01$. Plot a histogram showing the distribution of strategy frequencies once the system has reached equilibrium.
\end{enumerate}

\end{enumerate}


\begin{center}
{\Large \bf Computational Challenge 9
\\ \vskip5mm  }
\end{center}

In this challenge you will develop different agent based models.  These are representative of the
models people use in research.

It will be easiest to see what is going on if you animate  your simulations. While this is optional
in the first three challenges, we suggest that you do it using, for example, \verb|FuncAnimation| from \verb|matplotlib|.

\begin{enumerate}
\item[\bf Group 1]  In this challenge you will implement Schelling' model  of segregation.   
For a description of the model, see p. 108 of the book by Easley and Kleinberg.  As
the book notes, you will have several options in the implementation, but the results should
not differ.  I suggest using a double loop to go over rows and columns of the lattice sequentially, 
so that you can easily terminate the simulation if no agents have moved during one entire iteration.
Note that the system can also reach a dynamic equilibrium in which some of the agents will
keep moving, but the pattern is largely static.

The size of the domain is determined by the width and height. The ``empty ratio'' is the fraction of 
the domain that is initially unoccupied, and the similarity threshold determines the fraction of different neighbors above which 
an agent will move.  Set a maximal number of iterations, but also stop your simulations if your simulation
reaches a stable state. For the following you can use a 50 by 50, or larger domain.  The 
similarity ratio $r_{\text{sim}} = n_{\text{similar neighbors}}/n_{\text{neighbors}}$ is the fraction
of neighbors that have the same color as the agent.

\begin{enumerate}
\item  Start with \verb|empty_ratio = 0.2|, and increase the similarity threshold from 0.2 to 0.6.   
Run the simulation until the average similarity ratio, $r_{\text{sim}}$ converges (it may keep oscillating a little, so
add some tolerance).
Does convergence take longer for smaller or larger thresholds?  Explain why this could be.
\item  Start with \verb|similarity_threshold = 0.4|, and increase the \verb|empty_ratio| from 0.1 to 0.6 in increments of 0.1   
Run the simulation until convergence.  
Does convergence take longer for smaller or larger empty spaces?  Explain why this could be.
\item Run your simulations for  \verb|similarity_threshold| = 0.2--0.6 in increments of 0.1, and plot $r_{\text{sim}}$ over time.
How does  $r_{\text{sim}}$ relate to the similarity threshold?
\item Now assume that with some probability $r$ even satisfied agent moves.
That is there is a small probability $r$ that during a time step an agent moves to a random location, 
regardless of its neighborhood.  How does increasing $r$ impact your observations in part (b)?  
\item BONUS: Animate your simulation.
\end{enumerate}

\item[\bf Group 2] In this challenge you will implement a spatial Moran model of cancer, as 
discussed by N. Komarova, \emph{Bulletin of Mathematical Biology} (2006) 68: 1573--1599.
Assume that cells  are arranged in a regular grid, at locations $i = 0, 1, 2, \ldots , N$. 
In this compartment the total number of cells, $N,$ does not change, as each cell that dies
is replaced by a new cell. 

The simulation now follows the following steps
\begin{enumerate} 
\item[1.] A cell is chosen for death, and is removed from the population.
All cells are equally likely to die.
\item[2.] One of the two neighboring cells is chosen for reproduction. 
If the fitnesses of the two neighboring cells are $r_{\text{left}}$ and $r_{\text{right}}$, the 
probability that the left will reproduce is $r_{\text{left}}/(r_{\text{left}}+ r_{\text{right}})$, 
and the probability that the right will reproduce is $r_{\text{right}}/(r_{\text{left}}+ r_{\text{right}})$.
\item[3.] The
descendant of the dividing cell fills the empty spot created by the removal of the cell in step 1. 
\end{enumerate} 
 Implement this agent based model allowing the compartment size, $N,$ to be a variable parameter.  
You can use a compartment of size $N = 100$ or larger for the first part of the challenge. 

 \begin{enumerate}
 \item Starting with different initial positions for a single mutant cell, compute the fixation probabilities 
 of mutants with fitness $r = 0.98, 1$ (neutral), $,r = 1.02,$ and $1.1$.  You can choose a reasonable
 subsample of the grid for the initial position (say every fourth or so), depending on your compartment size.  Plot 
 fixation probability against position.  Are there edge effects, or is the probability uniform across the compartment? 
 \item Compute the fixation probability as a function of compartment size starting at $N = 20$.
 Start with a single mutant in the center position.    
 \item Compare the fixation probabilities in the previous example against fixation probabilities in 
 a regular Moran model with no spatial organization.  Remember that we have computed these 
 fixation probabilities in class. 
\item  Implement the following variation of the linear model discussed in Chapter 12 of Martin Nowak's book:
\begin{enumerate} 
\item[1.] A cell is chosen for reproduction according to its fitness, as in the non-spatial Moran model.
\item[2.] If the reproducing cell is left of center, it pushes all cells to the left of it by one position to the left, and the leftmost cell (at position $i =0$), 
is removed from the compartment.  If the cell is to the right of center it pushes all cells to the right of it by one position to the right.
The rightmost cell at position $i = N$ is removed. 
\item[3.] The descendant of the dividing cell fills the new empty spot created in step 2 to the left or the right of the dividing cell. 
\end{enumerate} 
Repeat part (a) of this challenge using this model.  Explain the difference between the results.
\item BONUS: Animate your simulation.
\end{enumerate}
You can find a master equation for the first version of the model in the paper by N. Komarova.
The master equation for the second version is more complicated.

\item[\bf Group 3] In this challenge you will implement the model described in the paper by I. Couzin, \emph{et al} we
discussed in class. To do so you will need to define vectors with agent positions, direction and speed.  
To update the position of the individuals, compute the direction of heading as described by equations 1-3 in
the paper. Read the methods carefully: There is small noise that is added to the movement in each step. 
Use the parameters given in the caption of Fig. 1.
\
 \begin{enumerate}
 \item Here you will examine the ability of a group to find a source of food.  Start your simulation with food
 located at the origin, and place the group in a circle of radius 2 at distance 50-100 units from the food.  You can
 start with a larger circle for larger populations.  An informed individual in the group will move in a direction
 ${\mathbf g}_i$ that points to the origin. To calibrate the time, first assume that all agents are informed,
 and strongly attracted to the food, \emph{ie} $\omega$ is large. You should see that the group reaches the food within about 
 100 time units. 
 \item Next recreate the plot in Fig 1a for populations of size 25, 50, and 100, and a different number of informed
 individuals.  Use $\omega = 0.5, 1,$ and $2$.  Terminate the simulations when all individuals are within a radius 10 of the food
 (success), or the simulation ran for more than 10000$s$ (failure).  Increment the number of informed individuals in steps of 
 approximately  5 \%, starting at  5 \%.  Use a number of simulations in each condition to estimate the chance of success.
 \item Assume that the population is running away from a predator instead of finding food.  The predator starts 10 units away from the centroid of the agent population, and always moves to the closest
 agent, regardless of distance. If the predator comes within 0.5 units of the agent, the agent is removed from the population.  In this case the vector  ${\mathbf g}_i$ points in the 
 direction opposite of the predator.  Assume that the speed of the predator  is 
 slightly larger than that of each agent,  $s_p > s$. Fix the population size at 25, and plot the survival time
  of the population as a function of the fraction of informed individuals, and the strength of interaction $\omega$.   
 \item BONUS: Animate your simulation.
 \end{enumerate}
 
\item[\bf Group 4]  In this challenge you will implement a spatial version of the rock--paper--scissors game. The agents
are placed on a lattice with periodic boundary conditions, and each agent uses one of three strategies.  We can 
thus think of the agents as belonging to one of three species.  Each cell (location) in the lattice is either empty or occupied by 
an agent of one particular species.

Start with populating the lattice with agents of each type, and leaving some cells empty.
The simulation proceeds by picking a cell uniformly at random in the lattice, and then picking a random neighboring cell.
If both cells are empty, or are occupied by agents of the same species, nothing happens.  Otherwise three things can happen:
\begin{enumerate}
\item[1.] If only one of the cells is occupied, the agent in the occupied cell will reproduce.
The descendant belongs to the same species, \emph{ie} uses the same strategy.
\item[2.]  If the two cells are occupied by different species, then with probability $p$ they fight. 
The agent with the loosing strategy dies, and their cell is vacated. 
\item[3.] With probability $1-p$, the agents in the two cells swap places.
\end{enumerate}

For this exercise it is best to animate your output.  You can use \verb|FuncAnimation| from \verb|matplotlib|, but there are other choices. 
Start with a 30 by 30 grid at least.

\begin{enumerate}
\item Illustrate what happens for high values of $p$, and then decrease $p$. You should see the formation of spiral structures that increase in size as mobility is increased.
\item Compute the time series for the frequency for each species, and plot them in time.  How does this change with $p$?
\item Assume that there is a higher rate of reactions in the upper left quadrant of the lattice, so that the probability of choosing a 
cell in this corner has probability is twice as likely as picking a cell from another quadrant.  How do the results in part (a) change?
\item BONUS: Extend this to five species by using the rules of the rock-paper-scissors-lizzard-spock game.
\end{enumerate}

\end{enumerate}



\end{document}
